% LaTeX Tables for Results Section
% Generated from Stage 4 Tables Analysis
% Date: 2025-01-31


% Table 1: Descriptive Statistics
\begin{table}[htbp]
\centering
\caption{Descriptive Statistics for Model Variants}
\label{tab:descriptive-statistics}
\begin{tabular}{lcccccc}
\toprule
\textbf{Model Variant} & \textbf{N} & \textbf{M} & \textbf{SD} & \textbf{95\% CI} & \textbf{Min} & \textbf{Max} \\
\midrule
Baseline           & 145 & 0.574 & 0.260 & [0.532, 0.616] & 0.0 & 1.0 \\
DPO-Synthetic      & 145 & 0.564 & 0.231 & [0.526, 0.602] & 0.0 & 1.0 \\
DPO-Hybrid         & 145 & 0.581 & 0.201 & [0.548, 0.614] & 0.0 & 1.0 \\
\bottomrule
\end{tabular}
\end{table}



% Table 2: Statistical Comparisons
\begin{table}[htbp]
\centering
\caption{Pairwise Statistical Comparisons Between Model Variants}
\label{tab:statistical-comparisons}
\begin{tabular}{lccccc}
\toprule
\textbf{Comparison} & \textbf{t} & \textbf{df} & \textbf{p} & \textbf{Cohen's d} & \textbf{95\% CI for d} \\
\midrule
Baseline vs DPO-Synthetic    & 0.412 & 144 & 0.681 & -0.040 & [-0.271, 0.190] \\
Baseline vs DPO-Hybrid       & -0.409 & 144 & 0.683 & 0.031 & [-0.199, 0.261] \\
DPO-Synthetic vs DPO-Hybrid  & -0.776 & 144 & 0.439 & 0.079 & [-0.151, 0.309] \\
\bottomrule
\end{tabular}
\begin{tablenotes}
\small
\item Note: All p-values > 0.05 indicate no statistically significant differences. All effect sizes are negligible (|d| < 0.2).
\end{tablenotes}
\end{table}



% Table 3: Model-Specific Performance
\begin{table}[htbp]
\centering
\caption{Individual Model Performance by Variant}
\label{tab:model-specific}
\begin{tabular}{lcccccc}
\toprule
\multirow{2}{*}{\textbf{Model}} & \multicolumn{2}{c}{\textbf{Baseline}} & \multicolumn{2}{c}{\textbf{DPO-Synthetic}} & \multicolumn{2}{c}{\textbf{DPO-Hybrid}} \\
\cmidrule(lr){2-3} \cmidrule(lr){4-5} \cmidrule(lr){6-7}
& \textbf{M} & \textbf{SD} & \textbf{M} & \textbf{Δ\%} & \textbf{M} & \textbf{Δ\%} \\
\midrule
M0001 & 0.591 & 0.234 & 0.571 & -3.4\% & 0.570 & -3.5\% \\
M0002 & 0.528 & 0.328 & 0.593 & +12.4\% & 0.615 & +16.7\% \\
M0003 & 0.535 & 0.195 & 0.555 & +3.8\% & 0.551 & +3.1\% \\
M0005 & 0.617 & 0.238 & 0.567 & -8.1\% & 0.553 & -10.4\% \\

\bottomrule
\end{tabular}
\begin{tablenotes}
\small
\item Note: Δ\% represents percentage change from baseline. M0001=TinyLlama, M0002=Vicuna, M0003=Phi-3, M0005=StableLM.
\end{tablenotes}
\end{table}



% Table 4: Category Analysis
\begin{table}[htbp]
\centering
\caption{Performance by Topic Category}
\label{tab:category-analysis}
\begin{tabular}{lcccccc}
\toprule
\multirow{2}{*}{\textbf{Category}} & \multicolumn{2}{c}{\textbf{Baseline}} & \multicolumn{2}{c}{\textbf{DPO-Synthetic}} & \multicolumn{2}{c}{\textbf{DPO-Hybrid}} \\
\cmidrule(lr){2-3} \cmidrule(lr){4-5} \cmidrule(lr){6-7}
& \textbf{M} & \textbf{N} & \textbf{M} & \textbf{Δ\%} & \textbf{M} & \textbf{Δ\%} \\
\midrule
Healthcare/Medical & 0.604 & 62 & 0.553 & -8.5\% & 0.557 & -7.7\% \\
Education/Youth & 0.525 & 62 & 0.522 & -0.6\% & 0.655 & +24.8\% \\
Environmental & 0.541 & 62 & 0.634 & +17.1\% & 0.586 & +8.2\% \\
Community/Social & 0.568 & 64 & 0.594 & +4.7\% & 0.528 & -7.1\% \\

\bottomrule
\end{tabular}
\begin{tablenotes}
\small
\item Note: Δ\% represents percentage change from baseline. Categories represent balanced segments of the evaluation data.
\end{tablenotes}
\end{table}



% Table 5: Methodology Validation
\begin{table}[htbp]
\centering
\caption{Methodology Validation: Predicted vs Actual Results}
\label{tab:methodology-validation}
\begin{tabular}{lccccc}
\toprule
\textbf{Comparison} & \textbf{Predicted d} & \textbf{Actual d} & \textbf{Within Range} & \textbf{Validation} \\
\midrule
Baseline vs DPO-Synthetic    & 0.5--0.7 & -0.040 & ✗ & FAIL \\
Baseline vs DPO-Hybrid       & 0.7--1.0 & 0.031 & ✗ & FAIL \\
DPO-Synthetic vs DPO-Hybrid  & 0.3--0.5 & 0.079 & ✗ & FAIL \\
\midrule
ANOVA η² Threshold & >0.06 & 0.001 & ✗ & FAIL \\
Expert Correlation & >0.80 & N/A & N/A & N/A \\
\midrule
	extbf{Overall Status} & \multicolumn{4}{c}{	extbf{FAIL}} \\
\bottomrule
\end{tabular}
\begin{tablenotes}
\small
\item Note: Methodology validation assesses whether empirical results match theoretical predictions. All effect size predictions and ANOVA threshold failed validation.
\end{tablenotes}
\end{table}


% Additional LaTeX packages required:
% \usepackage{booktabs}
% \usepackage{multirow}
% \usepackage{threeparttable}
